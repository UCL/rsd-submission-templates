\documentclass[]{scrartcl}
\usepackage[T1]{fontenc}
\usepackage{lmodern}
\usepackage{amssymb,amsmath}
\usepackage{ifxetex,ifluatex}
\usepackage[svgnames]{xcolor} % Enabling colors by their 'svgnames'
\usepackage{fixltx2e} % provides \textsubscript
% use microtype if available
\IfFileExists{microtype.sty}{\usepackage{microtype}}{}
% use upquote if available, for straight quotes in verbatim environments
\IfFileExists{upquote.sty}{\usepackage{upquote}}{}
\ifnum 0\ifxetex 1\fi\ifluatex 1\fi=0 % if pdftex
  \usepackage[utf8]{inputenc}
\else % if luatex or xelatex
  \usepackage{fontspec}
  \ifxetex
    \usepackage{xltxtra,xunicode}
  \fi
  \defaultfontfeatures{Mapping=tex-text,Scale=MatchLowercase}
  \newcommand{\euro}{€}
\fi
\usepackage{libertine}
\usepackage{framed}
\let\oldquote\quote
\let\endoldquote\endquote
\newcommand*\openquote{\makebox(25,-22){\scalebox{5}{\color{DarkGreen}``}}}
\newcommand*\closequote{\makebox(25,-22){\scalebox{5}{\color{DarkGreen}''}}}
\definecolor{verypalegreen}{RGB}{252,255,250}
\colorlet{shadecolor}{verypalegreen}

\makeatletter
\def\shadequote{\begin{snugshade}\begin{oldquote}\openquote}
\def\endshadequote{%
  \closequote\end{oldquote}\end{snugshade}}
\makeatother
  
\renewenvironment{quote}{\begin{shadequote}}{\end{shadequote}}
\usepackage{graphicx}
% We will generate all images so they have a width \maxwidth. This means
% that they will get their normal width if they fit onto the page, but
% are scaled down if they would overflow the margins.
\makeatletter
\def\maxwidth{\ifdim\Gin@nat@width>\linewidth\linewidth
\else\Gin@nat@width\fi}
\makeatother
\let\Oldincludegraphics\includegraphics
\renewcommand{\includegraphics}[1]{\Oldincludegraphics[width=\maxwidth]{#1}}
\ifxetex
  \usepackage[setpagesize=false, % page size defined by xetex
              unicode=false, % unicode breaks when used with xetex
              xetex]{hyperref}
\else
  \usepackage[unicode=true]{hyperref}
\fi
\hypersetup{breaklinks=true,
            bookmarks=true,
            pdfauthor={Prof. I. Principal, (PI, lecturer, Dept. Chemistry) Dr D. Post (Corresponding, d.post@ucl.ac.uk, RA, dept. Economics, Tel. 0203 549 0000) and G. Student (PhD Student, Pharmacology)},
            pdftitle={A Proposal for Important Work},
            colorlinks=true,
            urlcolor=blue,
            linkcolor=uclmidgreen,
            pdfborder={0 0 0}}
\urlstyle{same}  % don't use monospace font for urls
\setlength{\parindent}{0pt}
\setlength{\parskip}{6pt plus 2pt minus 1pt}
\setlength{\emergencystretch}{3em}  % prevent overfull lines
  
\usepackage{lettrine} % Package to accentuate the first letter of the text  
\usepackage{fix-cm}	 % Custom font sizes - used for the initial letter in the document

\definecolor{uclmidgreen}{RGB}{130,141,55}

\newcommand{\initial}[1]{ % Defines the command and style for the first letter
\lettrine[lines=3,lhang=0.3,nindent=0em]{
\color{DarkGreen}
{\textsf{#1}}}{}}

\usepackage{sectsty} % Enables custom section titles
\sectionfont{\color{uclmidgreen} \usefont{OT1}{phv}{b}{n}} % Change the font of all section commands
\subsectionfont{\color{DarkSeaGreen} \usefont{OT1}{phv}{b}{n}} % Change the font of all section commands

\usepackage{titling} % Allows custom title configuration 
\newcommand{\HorRule}{\color{DarkSeaGreen} \rule{\linewidth}{1pt}} % Defines the gold horizontal rule around the title

\pretitle{
\vspace{-30pt} \begin{flushleft} \HorRule \fontsize{20}{20} \usefont{OT1}{phv}{b}{n} \color{uclmidgreen} \selectfont} % Horizontal rule before the title

\posttitle{\par\end{flushleft}\begin{flushleft}\fontsize{15}{15} \usefont{OT1}{phv}{b}{n} \color{Black} \selectfont Submission to the RSD Call for Projects \end{flushleft}\vskip 0.5em} % Whitespace under the title and subtitle

\preauthor{\begin{flushleft}\large \lineskip 0.5em \usefont{OT1}{phv}{b}{sl} \color{uclmidgreen}} % Author font configuration
\title{A Proposal for Important Work}


\author{Prof.~I. Principal, (PI, lecturer, Dept. Chemistry) Dr D. Post
(Corresponding, d.post@ucl.ac.uk, RA, dept. Economics, Tel. 0203 549
0000) and G. Student (PhD Student, Pharmacology)}
\date{December 2013}

\postauthor{\footnotesize \usefont{OT1}{phv}{m}{sl} \color{Black} % Configuration for the institution name


University College London% Your institution

\par\end{flushleft}\HorRule} % Horizontal rule after the title
      

\usepackage{picture}
\usepackage{eso-pic}

\begin{document}
  \AddToShipoutPicture{\put(0,\dimexpr\paperheight-1.8cm){%
    \makebox[\paperwidth]{\Oldincludegraphics[width=\paperwidth,height=1.8cm]{bannermidgreen.pdf}}%
  }}

\maketitle

\section{Abstract}\label{abstract}

This should fit within the first page. It should cover the points you
consider most important in judging your proposal. While shortlisting,
the panel may read only this section. You should select points from
amongst the later sections to emphasise here.

\pagebreak

\section{Introduction}\label{introduction}

\subsection{Introduction to research
area}\label{introduction-to-research-area}

\begin{itemize}
\itemsep1pt\parskip0pt\parsep0pt
\item
  Assume a basic knowledge of the subject
\item
  As to a starting graduate student
\item
  Include key publications which can build a grounding in the field
\item
  Review the status of computational research in the area
\item
  References to any existing codes which complement or compete with the
  code being proposed for effort
\end{itemize}

\subsection{Introduction to research
group}\label{introduction-to-research-group}

\begin{itemize}
\itemsep1pt\parskip0pt\parsep0pt
\item
  Authors' track record in the field
\item
  Recent publications{[}1{]}

  \begin{itemize}
  \itemsep1pt\parskip0pt\parsep0pt
  \item
    Including those that will help the team understand the code to be
    worked on
  \end{itemize}
\item
  Computational experience of group

  \begin{itemize}
  \itemsep1pt\parskip0pt\parsep0pt
  \item
    Software training levels, languages, and competencies of staff
  \item
    Existing processes and tools used for organizing software
    development effort
  \end{itemize}
\end{itemize}

\subsection{Introduction to code to be worked
on}\label{introduction-to-code-to-be-worked-on}

\begin{itemize}
\itemsep1pt\parskip0pt\parsep0pt
\item
  If a brand new code, envisaged answers can be given
\item
  Overview of code purpose and use

  \begin{itemize}
  \itemsep1pt\parskip0pt\parsep0pt
  \item
    References to documentation or papers if available
  \end{itemize}
\item
  High level description of code structure and design approaches

  \begin{itemize}
  \itemsep1pt\parskip0pt\parsep0pt
  \item
    Suitable to help someone first looking at the code base
  \end{itemize}
\item
  Elements used, with references to web or research literature where
  appropriate:

  \begin{itemize}
  \itemsep1pt\parskip0pt\parsep0pt
  \item
    Languages
  \item
    Libraries
  \item
    Techniques and methods
  \item
    Algorithms
  \item
    Build tools
  \item
    Testing and deployment tools
  \end{itemize}
\item
  Important: provide \href{http://github.com/}{URL links} to existing
  code or email as attachment
\item
  Engineering status of existing code, with impact on research

  \begin{itemize}
  \itemsep1pt\parskip0pt\parsep0pt
  \item
    Degree of testing
  \item
    Incidence of crash or wrong-answer bugs
  \item
    Performance issues
  \item
    Readability and structural soundness
  \end{itemize}
\end{itemize}

\section{Suggested Objectives for The
Project}\label{suggested-objectives-for-the-project}

\begin{itemize}
\itemsep1pt\parskip0pt\parsep0pt
\item
  Suggested objectives for the project
\item
  Organised by priority
\item
  All must be justified in terms of research needs and impact

  \begin{itemize}
  \itemsep1pt\parskip0pt\parsep0pt
  \item
    Why is this feature or improvement needed?
  \item
    What will be the outcomes for the research group, for UCL, and for
    the field if this is implemented
  \end{itemize}
\item
  Final objectives will be agreed in collaboration with the team if
  selected

  \begin{itemize}
  \itemsep1pt\parskip0pt\parsep0pt
  \item
    Support for further requirements analysis will be provided as part
    of the project
  \item
    Effort is provided on a defined-effort not defined-outcome basis
  \end{itemize}
\item
  See
  \href{http://development.rc.ucl.ac.uk/termly-call/examples.html}{example
  suggested objectives}
\end{itemize}

\section{Impact of the project}\label{impact-of-the-project}

\subsection{Potential for use of software beyond originating
group}\label{potential-for-use-of-software-beyond-originating-group}

\begin{itemize}
\itemsep1pt\parskip0pt\parsep0pt
\item
  Development of software components, tools, insight or methods which
  could be of benefit to other research projects
\item
  Potential for receipt of otherwise unavailable research funding
\item
  Prevention of software falling into disuse or being forgotten
\item
  Development of skills within research group

  \begin{itemize}
  \itemsep1pt\parskip0pt\parsep0pt
  \item
    Transfer of software engineering knowledge to active
    computationally-focused PhD students and postdocs
  \end{itemize}
\item
  Unlocking potential for further development of code beyond engagement
\end{itemize}

\subsection{Sustainability of the
project}\label{sustainability-of-the-project}

\begin{itemize}
\itemsep1pt\parskip0pt\parsep0pt
\item
  How will the work project be maintained after RSDT free involvement
  ceases?
\item
  If by group staff, show that the group has the skills to maintain the
  code, or how these skills will be acquired (possibly with RSDT help).
\item
  If through subsequent paid work by RSDT, specify where resources will
  come from, either through existing funds or future grant applications.
\end{itemize}

\section{Justification for
application}\label{justification-for-application}

\subsection{Justification for use of RSDT
staff}\label{justification-for-use-of-rsdt-staff}

\begin{itemize}
\itemsep1pt\parskip0pt\parsep0pt
\item
  See appropriateness section in
  \href{http://development.rc.ucl.ac.uk/termly-call/selection.html}{selection
  criteria}
\item
  Why are research software developers required as opposed to general
  programmers?
\item
  Why can't existing research staff do this?
\end{itemize}

\subsection{Justification for use of free
project}\label{justification-for-use-of-free-project}

\begin{itemize}
\itemsep1pt\parskip0pt\parsep0pt
\item
  Existing chances of receiving funding for software development from
  other sources
\item
  Impact of project on those chances
\end{itemize}

1. Scientist A, Scholar T (2013) An example paper. The Journal of
Interesting Work 134: 1--2. Available: \url{http://example.com/paper}.
Accessed 01 January 2013.

\end{document}
